\PassOptionsToPackage{unicode=true}{hyperref} % options for packages loaded elsewhere
\PassOptionsToPackage{hyphens}{url}
%
\documentclass[]{article}
\usepackage{lmodern}
\usepackage{amssymb,amsmath}
\usepackage{ifxetex,ifluatex}
\usepackage{fixltx2e} % provides \textsubscript
\ifnum 0\ifxetex 1\fi\ifluatex 1\fi=0 % if pdftex
  \usepackage[T1]{fontenc}
  \usepackage[utf8]{inputenc}
  \usepackage{textcomp} % provides euro and other symbols
\else % if luatex or xelatex
  \usepackage{unicode-math}
  \defaultfontfeatures{Ligatures=TeX,Scale=MatchLowercase}
\fi
% use upquote if available, for straight quotes in verbatim environments
\IfFileExists{upquote.sty}{\usepackage{upquote}}{}
% use microtype if available
\IfFileExists{microtype.sty}{%
\usepackage[]{microtype}
\UseMicrotypeSet[protrusion]{basicmath} % disable protrusion for tt fonts
}{}
\IfFileExists{parskip.sty}{%
\usepackage{parskip}
}{% else
\setlength{\parindent}{0pt}
\setlength{\parskip}{6pt plus 2pt minus 1pt}
}
\usepackage{hyperref}
\hypersetup{
            pdftitle={Vision over Incidents and Claims Research Proposal},
            pdfauthor={Simardeep Kaur, Merve Sahin, Brayden Tang, Xugang (Kirk) Zhong},
            pdfborder={0 0 0},
            breaklinks=true}
\urlstyle{same}  % don't use monospace font for urls
\usepackage[margin=1in]{geometry}
\usepackage{graphicx,grffile}
\makeatletter
\def\maxwidth{\ifdim\Gin@nat@width>\linewidth\linewidth\else\Gin@nat@width\fi}
\def\maxheight{\ifdim\Gin@nat@height>\textheight\textheight\else\Gin@nat@height\fi}
\makeatother
% Scale images if necessary, so that they will not overflow the page
% margins by default, and it is still possible to overwrite the defaults
% using explicit options in \includegraphics[width, height, ...]{}
\setkeys{Gin}{width=\maxwidth,height=\maxheight,keepaspectratio}
\setlength{\emergencystretch}{3em}  % prevent overfull lines
\providecommand{\tightlist}{%
  \setlength{\itemsep}{0pt}\setlength{\parskip}{0pt}}
\setcounter{secnumdepth}{0}
% Redefines (sub)paragraphs to behave more like sections
\ifx\paragraph\undefined\else
\let\oldparagraph\paragraph
\renewcommand{\paragraph}[1]{\oldparagraph{#1}\mbox{}}
\fi
\ifx\subparagraph\undefined\else
\let\oldsubparagraph\subparagraph
\renewcommand{\subparagraph}[1]{\oldsubparagraph{#1}\mbox{}}
\fi

% set default figure placement to htbp
\makeatletter
\def\fps@figure{htbp}
\makeatother

\usepackage{graphicx} \usepackage{fancyhdr} \pagestyle{fancy} \setlength\headheight{28pt} \fancyhead[R]{\includegraphics[width=4cm]{logo.png}} \fancyfoot[R,RO]{}

\title{Vision over Incidents and Claims Research Proposal}
\author{Simardeep Kaur, Merve Sahin, Brayden Tang, Xugang (Kirk) Zhong}
\date{}

\begin{document}
\maketitle

\hypertarget{summary}{%
\section{Summary}\label{summary}}

In 2019/2020, TransLink has paid around 20 million in expenses for
insurance claims to ICBC. Additionally, their premium to ICBC has
increased by 200\% during the last five years. To investigate the
potential factors related to the increasing insurance cost and identify
the pattern among these predictors the research is proposing a
four-phase research plan, using statistical methods such as
zero-truncated regression, time series modeling, and mixed-effect
models. In a reproducible and interactive form, the research results
will be delivered to TransLink at the end.

\hypertarget{introduction}{%
\section{Introduction}\label{introduction}}

With the largest transit service area in Canada, TransLink is operating
more than 245 bus routes and 79 kilometers of rapid transit to meet the
transportation needs of 2.5 million people in Metro Vancouver as of the
end of 2018 (TransLink 2018). Legislation requires TransLink to carry a
\$1 million per occurrence liability policy on each of its revenue
vehicles and a \$200,000 per occurrence liability policy on each of its
non-revenue vehicles. Since 2014/2015, the premium paid to ICBC has
increased by over 200\% to cover onboard passenger injuries, cyclist
injuries, pedestrian injuries, and losses from collisions with third
party vehicles. For at-fault physical damage losses to its vehicles, the
premium paid to its own captive insurance company has increased by 33\%.
In response to soaring insurance costs and road safety concerns, this
research will investigate variables of interest that may influence the
frequency of transit incidents and claims. The research objectives are
to:

\begin{itemize}
\tightlist
\item
  Characterize current patterns of incident frequency in terms of
  variables related to drivers and vehicles
\item
  With the help of external data such as weather conditions and
  geographic data, examine how predictive these variables are of
  incidents/claims
\item
  Recommend actions for TransLink to reduce the incident frequency
  strategically
\item
  Identify research needs for future study
\end{itemize}

\hypertarget{data-product}{%
\section{Data Product}\label{data-product}}

A fully reproducible and interactive report will be delivered at the end
of the project. The report will give a visual representation of
relationships between the frequency of claims and specific variables
interactively. The project will use a fully reproducible data pipeline
so that the user can run the entire analysis using simple and
understandable make commands. A docker container for reproducibility on
any operating system will also be included.

\hypertarget{data-description}{%
\section{Data Description}\label{data-description}}

Below is a summary of the data sets that this project will be built on.

The key variables in each dataset that have the highest potential to
answer the core business problems are: - Bus model and length from the
``\emph{Bus Trip Information}'' dataset. - Experience and probation
status of the drivers from the ``\emph{Operator Occurrences}'' dataset.
- Loss location and claim description from the ``\emph{Collisions}''
dataset. - Loss date and more granular information about the claim
amount from the ``\emph{Claims}'' dataset.

\includegraphics{images/picture1.png}

\hypertarget{methodology}{%
\section{Methodology}\label{methodology}}

The research will focus on preventable, rather than non-preventable,
claims incidents. Non-preventable incidents are less likely to lend
themselves to easy interventions that could reduce insurance-related
costs. Preliminary analysis will focus on the incident occurrence rather
than incidence cost. This is because claims can remain open for an
extended time. which means costs cannot be assessed accurately.

Multiple separate analyses will be used to determine variables of
interest in predicting incident occurrence. Individual variables, such
as time, operator, location, and bus type, may require different
statistical or machine learning approaches to relate them to the
variable of interest. Through this analysis, features can then be
created that can be used to fit one composite model incorporating all
predictors simultaneously. To assess the predictive power of time and
weather on occurrences, the research will involve a time series
analysis, incorporating weather as an exogenous variable. In addition,
the predictive power of bus type and location are to be analyzed using
mixed-effect model approaches, due to the natural grouping of the data
(region and bus category). For simplicity, a linear mixed model will be
investigated first, using region and potentially bus category as a
random effect.

\includegraphics{images/picture3.png} Figure 1: Count of preventable
occurrences per asset vehicle year. Older vehicles appear to show a
greater number of incident occurrences. However, it is essential that
the number of each type of bus in use (in each year) is obtained to get
a fairer comparison so that rates can be calculated instead.

The operator dataset only contains operators with at least one
occurrence. Therefore, to account for this bias the research will
examine methods such as zero-truncated models or Bayesian regression
models. Finally, to find out whether certain descriptions and/or codes
exhibit similar loss experience, methods such as standard cluster
analyses will be deployed to identify potential groupings. Furthermore,
topic modeling for the manually written accident descriptions will be
used to further determine potential groupings or patterns in claim
behavior.

\includegraphics{images/picture4.png}

Figure 2: The incidents/year against operator experience. There is a
very clear correlation between operator experience and incident rate.

\hypertarget{timeline}{%
\section{Timeline}\label{timeline}}

The project will be completed with a four-phase research plan. After
refining the research objectives, the proposal is expected to be
composed during the first phrase. The second phase consists of three
tasks: data preprocessing and getting some preliminary results via
exploratory data analysis. Findings will be communicated directly to
TransLink to further explore and implement any possible feedback to
improve the analysis. Meanwhile, a Makefile is expected throughout the
project to establish a data pipeline. In the third phase, the final
predictive model will be developed, and the final interactive report
will be generated. With a docker container, the data pipeline will be
completed to make the project fully reproducible. The fourth phase is
dedicated to final reviews and completion of the project.

\includegraphics{images/picture2.png}

\hypertarget{reference}{%
\section{Reference}\label{reference}}

\end{document}
