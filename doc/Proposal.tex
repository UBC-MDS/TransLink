\PassOptionsToPackage{unicode=true}{hyperref} % options for packages loaded elsewhere
\PassOptionsToPackage{hyphens}{url}
%
\documentclass[]{article}
\usepackage{lmodern}
\usepackage{amssymb,amsmath}
\usepackage{ifxetex,ifluatex}
\usepackage{fixltx2e} % provides \textsubscript
\ifnum 0\ifxetex 1\fi\ifluatex 1\fi=0 % if pdftex
  \usepackage[T1]{fontenc}
  \usepackage[utf8]{inputenc}
  \usepackage{textcomp} % provides euro and other symbols
\else % if luatex or xelatex
  \usepackage{unicode-math}
  \defaultfontfeatures{Ligatures=TeX,Scale=MatchLowercase}
\fi
% use upquote if available, for straight quotes in verbatim environments
\IfFileExists{upquote.sty}{\usepackage{upquote}}{}
% use microtype if available
\IfFileExists{microtype.sty}{%
\usepackage[]{microtype}
\UseMicrotypeSet[protrusion]{basicmath} % disable protrusion for tt fonts
}{}
\IfFileExists{parskip.sty}{%
\usepackage{parskip}
}{% else
\setlength{\parindent}{0pt}
\setlength{\parskip}{6pt plus 2pt minus 1pt}
}
\usepackage{hyperref}
\hypersetup{
            pdftitle={Vision over Incidents and Claims Research Proposal},
            pdfauthor={Simardeep Kaur, Merve Sahin, Brayden Tang, Xugang (Kirk) Zhong},
            pdfborder={0 0 0},
            breaklinks=true}
\urlstyle{same}  % don't use monospace font for urls
\usepackage[margin=1in]{geometry}
\usepackage{graphicx,grffile}
\makeatletter
\def\maxwidth{\ifdim\Gin@nat@width>\linewidth\linewidth\else\Gin@nat@width\fi}
\def\maxheight{\ifdim\Gin@nat@height>\textheight\textheight\else\Gin@nat@height\fi}
\makeatother
% Scale images if necessary, so that they will not overflow the page
% margins by default, and it is still possible to overwrite the defaults
% using explicit options in \includegraphics[width, height, ...]{}
\setkeys{Gin}{width=\maxwidth,height=\maxheight,keepaspectratio}
\setlength{\emergencystretch}{3em}  % prevent overfull lines
\providecommand{\tightlist}{%
  \setlength{\itemsep}{0pt}\setlength{\parskip}{0pt}}
\setcounter{secnumdepth}{0}
% Redefines (sub)paragraphs to behave more like sections
\ifx\paragraph\undefined\else
\let\oldparagraph\paragraph
\renewcommand{\paragraph}[1]{\oldparagraph{#1}\mbox{}}
\fi
\ifx\subparagraph\undefined\else
\let\oldsubparagraph\subparagraph
\renewcommand{\subparagraph}[1]{\oldsubparagraph{#1}\mbox{}}
\fi

% set default figure placement to htbp
\makeatletter
\def\fps@figure{htbp}
\makeatother

\usepackage{graphicx} \usepackage{fancyhdr} \pagestyle{fancy} \setlength\headheight{28pt} \fancyhead[R]{\includegraphics[width=4cm]{logo.png}} \fancyfoot[R,RO]{}

\title{Vision over Incidents and Claims Research Proposal}
\author{Simardeep Kaur, Merve Sahin, Brayden Tang, Xugang (Kirk) Zhong}
\date{}

\begin{document}
\maketitle

\hypertarget{summary}{%
\section{Summary}\label{summary}}

To investigate the potential factors related to TransLink's increasing
insurance cost, the research is proposing a four-phase research plan,
deploying data science skills such as exploratory data analysis,
regression analysis and machine learning models to identify the
incidents/claims patterns and root causes among predictors such as
driver characteristics, vehicle characteristics and other third-party
predictors. A fully reproducible and interactive report generated via a
data pipeline is expected at the end.

\hypertarget{introduction}{%
\section{Introduction}\label{introduction}}

With the largest public transit service area in Canada, TransLink is
operating more than 245 bus routes, 79 kilometers of rapid transit, etc.
to meet the transportation need of 2.5 million people in Metro Vancouver
as of the end of 2018 (TransLink 2018). Legislatively, TransLink is
required to carry a \$1 million per occurrence liability policy on each
of its revenue vehicles and a \$200,000 per occurrence liability policy
on each of its non-revenue vehicles. Since 2014/2015, the premium paid
to ICBC has increased by over 200\% to cover onboard passenger injuries,
cyclist injuries, pedestrian injuries and losses from collisions with
third party vehicles. For at-fault physical damage losses to its
vehicles, the premium paid to its own captive insurance company has
increased by 33\%. In responding to the soaring insurance cost and road
safety concern, this research will investigate the factors that
influence the frequency and severity of transit incidents and claims,
identifying the incidents/claims pattern. The research objectives are
to:

\begin{itemize}
\tightlist
\item
  Characterize the current incidents in terms of driver-end predictors,
  vehicle-end predictors, and other predictors based on the
  incidents/claims, and operator data
\item
  With the help of external data such as weather conditions, and
  geographic data, examine the main predictors of incidents and
  potential incidents/claim pattern
\item
  Recommend actions for TransLink to reduce the insurance cost
  strategically
\item
  Identify research needs for future study
\end{itemize}

\hypertarget{data-product}{%
\section{Data Product}\label{data-product}}

A fully reproducible and interactive report will be emerging at the end
of the project. The report will give a visual representation of
relationships between the frequency or severity of claims and specific
variables interactively. The project will be done using a fully
reproducible data pipeline so that the user can run the entire analysis
using simple and understandable make commands. A docker container for
reproducibility on any operating system will also be included.

\hypertarget{data-description}{%
\section{Data Description}\label{data-description}}

\textbf{Bus Trip Information}: This dataset consists of individual trips
taken by busses over a span of five days in March 2020 with routes,
speeds, vehicle information, and metadata included.

\textbf{Operator Occurrences}: This dataset provides the number of
incidents (preventable and non-preventable) for all operators with at
least one incident in the past three years. Information regarding each
operator's characteristics is also included.

\textbf{Collisions}: This data set provides a detailed description of
both preventable and not preventable collisions that took place. Columns
describing the location and time of each collision are provided, along
with a brief description.

\textbf{Claims}: This dataset describes all occurrences as well as their
associated costs. It includes the location, time and description of the
occurrence, along with vehicle information.

\includegraphics{images/picture1.png}

\hypertarget{methodology}{%
\section{Methodology}\label{methodology}}

The research will mainly focus on preventable claims since the
non-preventable ones are less likely to lend themselves to easy
interventions used to reduce insurance-related costs. To model the
occurrences which do not yield any cost, we propose a hurdle model,
combing a binary classification model, predicting ``equal to zero'' and
``greater than zero'' over all occurrences with a regression model that
is fit only to non-zero cost occurrences. For simplicity, the research
will start with linear models. Alternatively, a different framework like
Tweedie regression (which allows for exact zeros) will also be
investigated. The model that predicts a held-out test set the best will
be used. In terms of bias in the operator dataset, the research will
examine methods such as zero-truncated models or Bayesian regression
models. To find out whether certain descriptions and/or codes exhibit
similar loss experience, methods such as standard cluster analyses will
be deployed to identify potential groupings. Furthermore, topic
modelling for the manually written accident descriptions will be used to
further determine potential groupings or patterns in claim behaviour.

\hypertarget{timeline}{%
\section{Timeline}\label{timeline}}

The project will be completed with a four-phase research plan. After
refining the research objectives, the proposal is expected to be
composed during the first phrase. The second phase consists of three
tasks: data preprocessing and getting some preliminary results via
exploratory data analysis. Findings will be communicated directly to
TransLink to further explore and implement any possible feedback to
improve the analysis. Meanwhile, a Makefile is expected throughout the
project to establish a data pipeline. In the third phase, the final
predictive model will be developed, and the final interactive report
will be generated. With a docker container, the data pipeline will be
completed to make the project fully reproducible. The fourth phase is
dedicated to final reviews and completion of the project.

\includegraphics{images/picture2.png}

\hypertarget{reference}{%
\section*{Reference}\label{reference}}
\addcontentsline{toc}{section}{Reference}

\hypertarget{refs}{}
\leavevmode\hypertarget{ref-TransLink}{}%
TransLink. 2018. \emph{2018 Accountability Report}. Vancouver, Canada:
TransLink.
\url{https://view.publitas.com/translink/2018-accountability-report/page/1}.

\end{document}
